\documentclass[9pt]{beamer}
\usepackage{tikz}
\usepackage{booktabs}

\definecolor{themecolor}{rgb}{.1, .1, .5}
\definecolor{darkgreen}{rgb}{.1, .1, .5}
\definecolor{myblue}{RGB}{0, 139, 188}

\def \mainroot{../../../}
\input{\mainroot/slides/fthb/local_config.tex}

\mode<presentation>{}
\usefonttheme{structuresmallcapsserif}
\setbeamerfont{footline}{size={\fontsize{10}{12}}}
\setbeamertemplate{footline}[frame number]{}
\setbeamertemplate{navigation symbols}{}

\title{FTHB model regressions}
\author{Berger, Cui, Turner, and Zwick \\ (DRAFT)}

\def \sdir{stata}
\def \mdir{matlab}


\begin{document}

\begin{frame}
\titlepage
\end{frame}

{
    \setbeamercolor{background canvas}{bg=themecolor}
\frame{
        \Large \color{white} \textbf{Model setup}
    \addtocounter{framenumber}{-1}
}
}

\begin{frame}{Defined parameters}
\textbf{Note:} The entire model is standardized to median household income
 in the 1998-2004 SCF (About ~\$67,000 in 2013 dollars)
\begin{itemize}
        \item $1-\alpha = 0.859:$ Cobb-Douglas parameter, share of expenditure
                in perishable consumption (i.e. $\alpha$ share in durables)
        \item $\gamma = 2:$ Intertemporal elasticity of substitution
        \item $r = 2.4\%:$ rate of return on the safe asset
        \item $r_{borrow} = r + 0.8\%$ interest rate on borrowing (if
                $q \leq (1-\theta)*h*p$)
        \item $\delta = 2.2\%:$ Depreciation rate of hdurable
        \item $F = 6\%:$ total fixed cost on adjusting durable stock
        \item $\underline{s} = 0.8:$ share of the fixed cost borne by the
                seller (i.e. she pays $\underline{s}F$)
        \item $\theta = 20\%$: Required down payment on durable
        \item $\rho_z = 0.91$: Persistence of AR(1) income process
        \item $\sigma_z = 0.20$: S.d. of shocks to income process
        \item $\epsilon = 2.5$: Price elasticity of supply for the representative
               housing firm
\end{itemize}
\end{frame}

\begin{frame}{Calibrated parameters}
        \textbf{Note:} Unlikely that all the parameters below will be
        calibrated.
\begin{itemize}
        \item $\beta = 0.915:$ Discount rate.
        \item $\phi = 0.26\%:$ Rental housing markup (added onto user cost of housing
                yields the rental price as a fraction of housing)
        \item $\phi_{ret} = 0.065\% : $ Rental housing markup in retirement.
        \item $h_{min} = 0.78:$ Minimum size for an owned house (no limits exist on
                renting)
        \item $\Xi = 2.00 :$ A lump sum transfer at retirement equal to a proportion
                of labour income before retirement
        \item $\Psi = 3.60:$ Multiplicative factor on bequest utility (seems large,
               but maybe bequests are also defined differently?)
       \item  $\omega:$ Disutility of rental housing ($=1$ for owned housing)
       \item  $\underline{b}:$ Reference value for bequests: affects marginal
               utility of a unit increase in bequests.
\end{itemize}
\end{frame}

\begin{frame}{Algorithmic details}
\begin{itemize}
        \item Search space over 120 uneven grid points for voluntary equity,
              $q = a + (1-\theta)*h*p$, 90 grid points for $h$
      \item 9 grid points for income process (Tauchen '86 discretization),
              with a range of $\pm 2.5$ the unconditional s.d
              of the AR(1)
      \item  38 working periods, 25 retirement periods. Correspond to ages
              22-84 on data
      \item A steady-state general equilibrium is found by minimizing the
              deviation between the \textbf{average} excess demand for housing
              (see Kaplan, Mitman, Violante, eq. 6) and the average new
              construction supply. \newline
              The minimizing price is found using Brent's method, with
              a liberal convergence threshold. However, the minimum deviation
              still usually reaches less than 1E-2.
\end{itemize}
\end{frame}

{
    \setbeamercolor{background canvas}{bg=themecolor}
\frame{
        \Large \color{white} \textbf{Regression Tables, Baseline Model}
    \addtocounter{framenumber}{-1}
}
}

\begin{frame}{Policy downscaling, inframarginal}
   \footnotesize \begin{table}[htbp]\centering
\def\sym#1{\ifmmode^{#1}\else\(^{#1}\)\fi}
\caption{Policy downscaling, inframarginal}
\begin{tabular}{l*{2}{c}}
\hline\hline
                    &\multicolumn{2}{c}{Chg in house size under policy vs. previous rental housing}\\\cmidrule(lr){2-3}
                    &\multicolumn{1}{c}{(1)}         &\multicolumn{1}{c}{(2)}         \\
\hline
Income value, T (policy period)&     -0.0604\sym{***}&       0.148\sym{***}\\
                    &   (0.00151)         &   (0.00182)         \\
Income shock received, period T-1&       0.157\sym{***}&                     \\
                    &  (0.000975)         &                     \\
Existing Assets/Loans&                     &     -0.0600\sym{***}\\
                    &                     &   (0.00237)         \\
Assets change received, period T-1&                     &      -1.472\sym{***}\\
                    &                     &    (0.0217)         \\
Constant            &      0.0603\sym{***}&     -0.0711\sym{***}\\
                    &   (0.00405)         &   (0.00667)         \\
\hline
Observations        &        6505         &        6505         \\
\hline\hline
\multicolumn{3}{l}{\footnotesize Standard errors in parentheses}\\
\multicolumn{3}{l}{\footnotesize \sym{*} \(p<0.05\), \sym{**} \(p<0.01\), \sym{***} \(p<0.001\)}\\
\end{tabular}
\end{table}

\end{frame}

\begin{frame}{Policy downscaling, marginal}
   \footnotesize \begin{table}[htbp]\centering
\def\sym#1{\ifmmode^{#1}\else\(^{#1}\)\fi}
\caption{Policy downscaling, Marginal}
\begin{tabular}{l*{2}{c}}
\hline\hline
                    &\multicolumn{2}{c}{Chg in house size under policy vs. previous rental housing}\\\cmidrule(lr){2-3}
                    &\multicolumn{1}{c}{(1)}         &\multicolumn{1}{c}{(2)}         \\
\hline
Income value, T (policy period)&      -0.231\sym{***}&     -0.0926\sym{***}\\
                    &   (0.00140)         &   (0.00258)         \\
Income shock received, period T-1&       0.215\sym{***}&                     \\
                    &   (0.00104)         &                     \\
Existing Assets/Loans&                     &      -0.169\sym{***}\\
                    &                     &   (0.00213)         \\
Assets change received, period T-1&                     &      -2.269\sym{***}\\
                    &                     &    (0.0322)         \\
Constant            &      0.0936\sym{***}&     -0.0706\sym{***}\\
                    &   (0.00316)         &   (0.00533)         \\
\hline
Observations        &       12856         &       12856         \\
\hline\hline
\multicolumn{3}{l}{\footnotesize Standard errors in parentheses}\\
\multicolumn{3}{l}{\footnotesize \sym{*} \(p<0.05\), \sym{**} \(p<0.01\), \sym{***} \(p<0.001\)}\\
\end{tabular}
\end{table}

\end{frame}

\begin{frame}{Timing margin, marginal}
   \tiny \begin{table}[htbp]\centering
\def\sym#1{\ifmmode^{#1}\else\(^{#1}\)\fi}
\caption{Timing margin, marginal}
\begin{tabular}{p{1.2in}l*{4}{c}}
\hline\hline
                    &\multicolumn{5}{c}{Chg in years purchase pulled forward}                                                     \\\cmidrule(lr){2-6}
                    &\multicolumn{1}{c}{(1)}         &\multicolumn{1}{c}{(2)}         &\multicolumn{1}{c}{(3)}         &\multicolumn{1}{c}{(4)}         &\multicolumn{1}{c}{(5)}         \\
\hline
Income value, T (policy period)&      -2.159\sym{***}&                     &      -2.392\sym{***}&      -1.105\sym{***}&      -1.732\sym{***}\\
                    &    (0.0922)         &                     &    (0.0894)         &     (0.104)         &     (0.168)         \\
Income shock received, period T-1&      0.0556         &                     &                     &      0.0131         &       0.215\sym{*}  \\
                    &    (0.0568)         &                     &                     &    (0.0571)         &    (0.0964)         \\
Existing Assets/Loans&                     &       0.402\sym{***}&      -0.295\sym{***}&                     &                     \\
                    &                     &    (0.0697)         &    (0.0722)         &                     &                     \\
Assets change received, period T-1&                     &      -1.925         &       0.207         &                     &                     \\
                    &                     &     (1.107)         &     (1.073)         &                     &                     \\
Income shock received, period T+1&                     &                     &                     &      -1.234\sym{***}&      -0.677\sym{***}\\
                    &                     &                     &                     &    (0.0896)         &     (0.160)         \\
Income shock received, period T+2&                     &                     &                     &      -0.382\sym{***}&      0.0858         \\
                    &                     &                     &                     &    (0.0962)         &     (0.172)         \\
Income shock received, period T+3&                     &                     &                     &      -0.301\sym{***}&      -0.335\sym{*}  \\
                    &                     &                     &                     &    (0.0761)         &     (0.135)         \\
Age Minus 20 $\times$ Income value, T (policy period)&                     &                     &                     &                     &      0.0452\sym{***}\\
                    &                     &                     &                     &                     &    (0.0107)         \\
Age Minus 20 $\times$ Income shock received, period T-1&                     &                     &                     &                     &     -0.0145\sym{*}  \\
                    &                     &                     &                     &                     &   (0.00639)         \\
Age Minus 20 $\times$ Income shock received, period T+1&                     &                     &                     &                     &     -0.0384\sym{***}\\
                    &                     &                     &                     &                     &   (0.00922)         \\
Age Minus 20 $\times$ Income shock received, period T+2&                     &                     &                     &                     &     -0.0318\sym{**} \\
                    &                     &                     &                     &                     &   (0.00993)         \\
Age Minus 20 $\times$ Income shock received, period T+3&                     &                     &                     &                     &     0.00200         \\
                    &                     &                     &                     &                     &   (0.00780)         \\
Constant            &       3.610\sym{***}&       2.964\sym{***}&       3.554\sym{***}&       5.319\sym{***}&       4.633\sym{***}\\
                    &     (0.165)         &     (0.167)         &     (0.163)         &     (0.176)         &     (0.198)         \\
\hline
Observations        &       10119         &       10119         &       10119         &        9535         &        9535         \\
\hline\hline
\multicolumn{6}{l}{\footnotesize Standard errors in parentheses}\\
\multicolumn{6}{l}{\footnotesize \sym{*} \(p<0.05\), \sym{**} \(p<0.01\), \sym{***} \(p<0.001\)}\\
\end{tabular}
\end{table}

\end{frame}

\begin{frame}{Extensive margin reversion (T+1), inframarginal}
   \tiny \begin{table}[htbp]\centering
\def\sym#1{\ifmmode^{#1}\else\(^{#1}\)\fi}
\caption{Extensive margin reversion, inframarginal}
\begin{tabular}{p{0.8in}l*{5}{c}}
\hline\hline
                    &\multicolumn{1}{c}{(1)}         &                    &\multicolumn{1}{c}{(2)}         &                     &\multicolumn{1}{c}{(3)}         &                     \\
                    &  Adjustment         &      Rental         &  Adjustment         &      Rental         &  Adjustment         &      Rental         \\
\hline
Income shock received, period T+1&       3.331\sym{***}&      -1.748\sym{***}&       3.324\sym{***}&      -1.742\sym{***}&                     &                     \\
                    &     (0.203)         &     (0.131)         &     (0.203)         &     (0.134)         &                     &                     \\
Income value, T (policy period)&      -0.604\sym{***}&      -0.670\sym{***}&      -0.738\sym{***}&      -0.534\sym{**} &       0.162         &      -1.332\sym{***}\\
                    &     (0.164)         &     (0.172)         &     (0.147)         &     (0.205)         &    (0.0891)         &     (0.169)         \\
Income shock received, period T-1&      -0.114         &    -0.00884         &                     &                     &      0.0149         &     -0.0149         \\
                    &     (0.111)         &     (0.128)         &                     &                     &    (0.0817)         &     (0.120)         \\
Existing Assets/Loans&                     &                     &      -0.206         &       0.284         &                     &                     \\
                    &                     &                     &     (0.272)         &     (0.151)         &                     &                     \\
Assets change received, period T-1&                     &                     &      -0.492         &       0.428         &                     &                     \\
                    &                     &                     &     (1.970)         &     (1.670)         &                     &                     \\
Chg in Consumption vs. counterfactual, policy period&                     &                     &                     &                     &     -0.0524         &       2.533         \\
                    &                     &                     &                     &                     &     (2.105)         &     (1.742)         \\
Constant            &      -10.57\sym{***}&      -2.529\sym{***}&      -10.25\sym{***}&      -2.852\sym{***}&      -4.539\sym{***}&      -3.136\sym{***}\\
                    &     (0.753)         &     (0.558)         &     (0.701)         &     (0.582)         &     (0.478)         &     (0.506)         \\
\hline
Observations        &        6548         &                     &        6548         &                     &        6794         &                     \\
\hline\hline
\multicolumn{7}{l}{\footnotesize Standard errors in parentheses}\\
\multicolumn{7}{l}{\footnotesize \sym{*} \(p<0.05\), \sym{**} \(p<0.01\), \sym{***} \(p<0.001\)}\\
\end{tabular}
\end{table}

\end{frame}

\begin{frame}{Extensive margin reversion (T+1), marginal}
   \tiny \begin{table}[htbp]\centering
\def\sym#1{\ifmmode^{#1}\else\(^{#1}\)\fi}
\caption{Extensive margin reversion, marginal}
\begin{tabular}{p{0.8in}l*{5}{c}}
\hline\hline
                    &\multicolumn{1}{c}{(1)}         &                     &\multicolumn{1}{c}{(2)}         &                     &\multicolumn{1}{c}{(3)}         &                     \\
                    &  Adjustment         &      Rental         &  Adjustment         &      Rental         &  Adjustment         &      Rental         \\
\hline
Income shock received, period T+1&       6.587\sym{***}&      -2.782\sym{***}&       6.595\sym{***}&      -2.991\sym{***}&                     &                     \\
                    &     (0.374)         &    (0.0859)         &     (0.375)         &    (0.0925)         &                     &                     \\
Income value, T (policy period)&       0.172         &      -1.664\sym{***}&       0.130         &      -0.570\sym{***}&       1.686\sym{***}&      -3.157\sym{***}\\
                    &     (0.261)         &    (0.0903)         &     (0.239)         &     (0.111)         &     (0.166)         &    (0.0769)         \\
Income shock received, period T-1&    0.000830         &       0.235\sym{***}&                     &                     &     -0.0771         &       0.167\sym{**} \\
                    &     (0.158)         &    (0.0671)         &                     &                     &    (0.0953)         &    (0.0591)         \\
Existing Assets/Loans&                     &                     &      -0.250         &       0.518\sym{***}&                     &                     \\
                    &                     &                     &     (0.388)         &    (0.0589)         &                     &                     \\
Assets change received, period T-1&                     &                     &       2.743         &      -2.813\sym{*}  &                     &                     \\
                    &                     &                     &     (4.792)         &     (1.114)         &                     &                     \\
Chg in Consumption vs. counterfactual, policy period&                     &                     &                     &                     &      -1.154         &       2.465\sym{***}\\
                    &                     &                     &                     &                     &     (0.854)         &     (0.259)         \\
Constant            &      -17.05\sym{***}&      -2.094\sym{***}&      -17.23\sym{***}&      -2.620\sym{***}&      -4.308\sym{***}&      -3.512\sym{***}\\
                    &     (0.994)         &     (0.259)         &     (1.007)         &     (0.289)         &     (0.304)         &     (0.229)         \\
\hline
Observations        &       12645         &                     &       12645         &                     &       12856         &                     \\
\hline\hline
\multicolumn{7}{l}{\footnotesize Standard errors in parentheses}\\
\multicolumn{7}{l}{\footnotesize \sym{*} \(p<0.05\), \sym{**} \(p<0.01\), \sym{***} \(p<0.001\)}\\
\end{tabular}
\end{table}

\end{frame}

\begin{frame}{Extensive margin reversion (T+2), inframarginal}
   \tiny \begin{table}[htbp]\centering
\def\sym#1{\ifmmode^{#1}\else\(^{#1}\)\fi}
\caption{Extensive margin reversion, inframarginal}
\begin{tabular}{p{0.8in}l*{5}{c}}
\hline\hline
                    &\multicolumn{1}{c}{(1)}         &                     &\multicolumn{1}{c}{(2)}         &                     &\multicolumn{1}{c}{(3)}         &                     \\
                    &  Adjustment         &      Rental         &  Adjustment         &      Rental         &  Adjustment         &      Rental         \\
\hline
Income shock received, period T+1&       0.437\sym{***}&      -0.445\sym{**} &       0.422\sym{***}&      -0.439\sym{**} &                     &                     \\
                    &     (0.119)         &     (0.146)         &     (0.120)         &     (0.148)         &                     &                     \\
Income shock received, period T+2&       2.346\sym{***}&      -2.195\sym{***}&       2.357\sym{***}&      -2.226\sym{***}&                     &                     \\
                    &     (0.123)         &     (0.145)         &     (0.124)         &     (0.147)         &                     &                     \\
Income value, T (policy period)&      -0.903\sym{***}&      -0.785\sym{***}&      -0.957\sym{***}&      -0.275         &     -0.0708         &      -1.259\sym{***}\\
                    &     (0.107)         &     (0.160)         &    (0.0970)         &     (0.171)         &    (0.0659)         &     (0.126)         \\
Income shock received, period T-1&      0.0608         &       0.293\sym{**} &                     &                     &      0.0664         &       0.267\sym{**} \\
                    &    (0.0663)         &     (0.107)         &                     &                     &    (0.0510)         &    (0.0885)         \\
Existing Assets/Loans&                     &                     &      -0.499\sym{**} &       0.334\sym{**} &                     &                     \\
                    &                     &                     &     (0.157)         &     (0.122)         &                     &                     \\
Assets change received, period T-1&                     &                     &      0.0877         &      -1.395         &                     &                     \\
                    &                     &                     &     (1.101)         &     (1.398)         &                     &                     \\
Chg in Consumption vs. counterfactual, period T+1&                     &                     &                     &                     &       4.791         &       9.388\sym{***}\\
                    &                     &                     &                     &                     &     (3.160)         &     (1.950)         \\
Chg in Consumption vs. counterfactual, policy period&                     &                     &                     &                     &      -1.027         &       3.062\sym{*}  \\
                    &                     &                     &                     &                     &     (1.135)         &     (1.425)         \\
Constant            &      -5.969\sym{***}&      -2.333\sym{***}&      -6.036\sym{***}&      -2.994\sym{***}&      -2.509\sym{***}&      -2.756\sym{***}\\
                    &     (0.347)         &     (0.492)         &     (0.334)         &     (0.513)         &     (0.253)         &     (0.407)         \\
\hline
Observations        &        6416         &                     &        6416         &                     &        6788         &                     \\
\hline\hline
\multicolumn{7}{l}{\footnotesize Standard errors in parentheses}\\
\multicolumn{7}{l}{\footnotesize \sym{*} \(p<0.05\), \sym{**} \(p<0.01\), \sym{***} \(p<0.001\)}\\
\end{tabular}
\end{table}

\end{frame}

\begin{frame}{Extensive margin reversion (T+2), marginal}
   \tiny \begin{table}[htbp]\centering
\def\sym#1{\ifmmode^{#1}\else\(^{#1}\)\fi}
\caption{Extensive margin reversion, marginal}
\begin{tabular}{p{0.8in}l*{5}{c}}
\hline\hline
                    &\multicolumn{1}{c}{(1)}         &                     &\multicolumn{1}{c}{(2)}         &                     &\multicolumn{1}{c}{(3)}         &                     \\
                    &  Adjustment         &      Rental         &  Adjustment         &      Rental         &  Adjustment         &      Rental         \\
\hline
Income shock received, period T+1&       0.259\sym{**} &      -0.932\sym{***}&       0.288\sym{**} &      -0.974\sym{***}&                     &                     \\
                    &    (0.0974)         &    (0.0857)         &    (0.0973)         &    (0.0869)         &                     &                     \\
Income shock received, period T+2&       3.232\sym{***}&      -2.677\sym{***}&       3.239\sym{***}&      -2.756\sym{***}&                     &                     \\
                    &     (0.118)         &    (0.0896)         &     (0.118)         &    (0.0921)         &                     &                     \\
Income value, T (policy period)&      -0.241         &      -1.640\sym{***}&       0.331\sym{**} &      -0.750\sym{***}&       0.901\sym{***}&      -2.492\sym{***}\\
                    &     (0.125)         &    (0.0946)         &     (0.114)         &     (0.106)         &    (0.0887)         &    (0.0660)         \\
Income shock received, period T-1&       0.263\sym{***}&       0.269\sym{***}&                     &                     &       0.168\sym{**} &       0.250\sym{***}\\
                    &    (0.0687)         &    (0.0617)         &                     &                     &    (0.0538)         &    (0.0489)         \\
Existing Assets/Loans&                     &                     &       0.435\sym{***}&       0.293\sym{***}&                     &                     \\
                    &                     &                     &     (0.101)         &    (0.0587)         &                     &                     \\
Assets change received, period T-1&                     &                     &      -3.318\sym{*}  &      -3.419\sym{***}&                     &                     \\
                    &                     &                     &     (1.581)         &     (1.018)         &                     &                     \\
Chg in Consumption vs. counterfactual, period T+1&                     &                     &                     &                     &      -2.698\sym{***}&      -1.916\sym{***}\\
                    &                     &                     &                     &                     &     (0.450)         &     (0.269)         \\
Chg in Consumption vs. counterfactual, policy period&                     &                     &                     &                     &       0.639         &       1.172\sym{***}\\
                    &                     &                     &                     &                     &     (0.374)         &     (0.226)         \\
Constant            &      -7.261\sym{***}&      -2.726\sym{***}&      -7.511\sym{***}&      -2.980\sym{***}&      -2.002\sym{***}&      -2.927\sym{***}\\
                    &     (0.268)         &     (0.254)         &     (0.267)         &     (0.264)         &     (0.147)         &     (0.193)         \\
\hline
Observations        &       12385         &                     &       12385         &                     &       12855         &                     \\
\hline\hline
\multicolumn{7}{l}{\footnotesize Standard errors in parentheses}\\
\multicolumn{7}{l}{\footnotesize \sym{*} \(p<0.05\), \sym{**} \(p<0.01\), \sym{***} \(p<0.001\)}\\
\end{tabular}
\end{table}

\end{frame}
\begin{frame}{Extensive margin reversion (T+3), inframarginal}
   \tiny \begin{table}[htbp]\centering
\def\sym#1{\ifmmode^{#1}\else\(^{#1}\)\fi}
\caption{Extensive margin reversion, inframarginal}
\begin{tabular}{p{0.8in}l*{5}{c}}
\hline\hline
                    &\multicolumn{1}{c}{(1)}         &                     &\multicolumn{1}{c}{(2)}         &                     &\multicolumn{1}{c}{(3)}         &                     \\
                    &  Adjustment         &      Rental         &  Adjustment         &      Rental         &  Adjustment         &      Rental         \\
\hline
Income shock received, period T+1&       0.162         &     -0.0271         &       0.153         &    0.000139         &                     &                     \\
                    &    (0.0989)         &     (0.142)         &    (0.0993)         &     (0.141)         &                     &                     \\
Income shock received, period T+2&       0.738\sym{***}&      -0.726\sym{***}&       0.740\sym{***}&      -0.718\sym{***}&                     &                     \\
                    &     (0.116)         &     (0.155)         &     (0.116)         &     (0.156)         &                     &                     \\
Income shock received, period T+3&       1.611\sym{***}&      -2.576\sym{***}&       1.623\sym{***}&      -2.625\sym{***}&                     &                     \\
                    &    (0.0994)         &     (0.158)         &     (0.100)         &     (0.160)         &                     &                     \\
Income value, T (policy period)&      -0.711\sym{***}&      -0.755\sym{***}&      -0.834\sym{***}&      -0.155         &     -0.0825         &      -0.819\sym{***}\\
                    &    (0.0803)         &     (0.146)         &    (0.0748)         &     (0.121)         &    (0.0500)         &     (0.103)         \\
Income shock received, period T-1&     -0.0302         &       0.423\sym{***}&                     &                     &     0.00357         &       0.276\sym{***}\\
                    &    (0.0516)         &    (0.0940)         &                     &                     &    (0.0405)         &    (0.0702)         \\
Existing Assets/Loans&                     &                     &      -0.472\sym{***}&       0.126         &                     &                     \\
                    &                     &                     &     (0.112)         &     (0.117)         &                     &                     \\
Assets change received, period T-1&                     &                     &      -0.983         &      -3.768\sym{**} &                     &                     \\
                    &                     &                     &     (0.832)         &     (1.196)         &                     &                     \\
Chg in Consumption vs. counterfactual, period T+1&                     &                     &                     &                     &       1.231         &      -9.328\sym{***}\\
                    &                     &                     &                     &                     &     (2.907)         &     (2.536)         \\
Chg in Consumption vs. counterfactual, period T+2&                     &                     &                     &                     &       2.630         &       17.43\sym{***}\\
                    &                     &                     &                     &                     &     (2.700)         &     (1.791)         \\
Chg in Consumption vs. counterfactual, policy period&                     &                     &                     &                     &      -2.036\sym{*}  &       0.435         \\
                    &                     &                     &                     &                     &     (0.869)         &     (1.079)         \\
Constant            &      -4.369\sym{***}&      -2.071\sym{***}&      -4.221\sym{***}&      -2.661\sym{***}&      -1.729\sym{***}&      -2.486\sym{***}\\
                    &     (0.261)         &     (0.408)         &     (0.248)         &     (0.412)         &     (0.193)         &     (0.324)         \\
\hline
Observations        &        6294         &                     &        6294         &                     &        6779         &                     \\
\hline\hline
\multicolumn{7}{l}{\footnotesize Standard errors in parentheses}\\
\multicolumn{7}{l}{\footnotesize \sym{*} \(p<0.05\), \sym{**} \(p<0.01\), \sym{***} \(p<0.001\)}\\
\end{tabular}
\end{table}

\end{frame}

\begin{frame}{Extensive margin reversion (T+3), marginal}
   \tiny \begin{table}[htbp]\centering
\def\sym#1{\ifmmode^{#1}\else\(^{#1}\)\fi}
\caption{Extensive margin reversion, marginal}
\begin{tabular}{p{0.8in}l*{5}{c}}
\hline\hline
                    &\multicolumn{1}{c}{(1)}         &                     &\multicolumn{1}{c}{(2)}         &                     &\multicolumn{1}{c}{(3)}         &                     \\
                    &  Adjustment         &      Rental         &  Adjustment         &      Rental         &  Adjustment         &      Rental         \\
\hline
Income shock received, period T+1&       0.164\sym{*}  &      -0.259\sym{**} &       0.220\sym{**} &      -0.272\sym{**} &                     &                     \\
                    &    (0.0819)         &    (0.0860)         &    (0.0827)         &    (0.0860)         &                     &                     \\
Income shock received, period T+2&       0.669\sym{***}&      -1.072\sym{***}&       0.662\sym{***}&      -1.092\sym{***}&                     &                     \\
                    &    (0.0907)         &    (0.0979)         &    (0.0918)         &    (0.0986)         &                     &                     \\
Income shock received, period T+3&       2.464\sym{***}&      -2.744\sym{***}&       2.526\sym{***}&      -2.754\sym{***}&                     &                     \\
                    &    (0.0907)         &    (0.0959)         &    (0.0920)         &    (0.0966)         &                     &                     \\
Income value, T (policy period)&     -0.0691         &      -1.639\sym{***}&       0.525\sym{***}&      -1.044\sym{***}&       0.670\sym{***}&      -1.957\sym{***}\\
                    &    (0.0990)         &    (0.0989)         &    (0.0946)         &     (0.107)         &    (0.0695)         &    (0.0587)         \\
Income shock received, period T-1&       0.139\sym{**} &       0.185\sym{**} &                     &                     &      0.0440         &       0.210\sym{***}\\
                    &    (0.0535)         &    (0.0603)         &                     &                     &    (0.0416)         &    (0.0441)         \\
Existing Assets/Loans&                     &                     &       0.480\sym{***}&       0.177\sym{**} &                     &                     \\
                    &                     &                     &    (0.0763)         &    (0.0612)         &                     &                     \\
Assets change received, period T-1&                     &                     &      -4.203\sym{***}&      -2.607\sym{*}  &                     &                     \\
                    &                     &                     &     (1.202)         &     (1.019)         &                     &                     \\
Chg in Consumption vs. counterfactual, period T+1&                     &                     &                     &                     &       0.292         &      -3.108\sym{***}\\
                    &                     &                     &                     &                     &     (0.366)         &     (0.271)         \\
Chg in Consumption vs. counterfactual, period T+2&                     &                     &                     &                     &      -3.492\sym{***}&       0.985\sym{**} \\
                    &                     &                     &                     &                     &     (0.462)         &     (0.317)         \\
Chg in Consumption vs. counterfactual, policy period&                     &                     &                     &                     &       1.348\sym{***}&       0.546\sym{**} \\
                    &                     &                     &                     &                     &     (0.292)         &     (0.211)         \\
Constant            &      -4.553\sym{***}&      -3.181\sym{***}&      -4.775\sym{***}&      -3.318\sym{***}&      -0.738\sym{***}&      -2.389\sym{***}\\
                    &     (0.177)         &     (0.247)         &     (0.180)         &     (0.251)         &     (0.109)         &     (0.170)         \\
\hline
Observations        &       12129         &                     &       12129         &                     &       12848         &                     \\
\hline\hline
\multicolumn{7}{l}{\footnotesize Standard errors in parentheses}\\
\multicolumn{7}{l}{\footnotesize \sym{*} \(p<0.05\), \sym{**} \(p<0.01\), \sym{***} \(p<0.001\)}\\
\end{tabular}
\end{table}

\end{frame}
\end{document}
