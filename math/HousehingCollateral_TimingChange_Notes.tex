%% LyX 2.3.0 created this file.  For more info, see http://www.lyx.org/.
%% Do not edit unless you really know what you are doing.
\documentclass[english]{article}
\usepackage[T1]{fontenc}
\usepackage[latin9]{inputenc}
\usepackage{rotfloat}
\usepackage{amsmath}
\usepackage{amsthm}
\usepackage{graphicx}

\makeatletter

%%%%%%%%%%%%%%%%%%%%%%%%%%%%%% LyX specific LaTeX commands.
%% Because html converters don't know tabularnewline
\providecommand{\tabularnewline}{\\}

%%%%%%%%%%%%%%%%%%%%%%%%%%%%%% Textclass specific LaTeX commands.
\theoremstyle{remark}
\newtheorem*{claim*}{\protect\claimname}

\makeatother

\usepackage{babel}
\providecommand{\claimname}{Claim}

\begin{document}
A household is born with $1$ unit of housing, $h_{-1}=1$ and lives
for two periods. A unit of house is worth $p$. The household must
pay a fraction $(1-\gamma)ph_{-1}$ in period $0$, meaning that only
a fraction $\gamma$ of the house value was paid ``yesterday'' (normalization).
The household has income $w_{0}$ and $w_{1}$ in periods $0$ and
$1$ respectively. The agent only values housing in period $1$, and
consumption in both periods. The utility function is
\[
u(c_{0},c_{1},v)=c_{0}+c_{1}+v(h_{1})
\]
The household's budget constraints are
\[
c_{0}+\gamma ph_{0}\mathcal{I}(h_{0}\neq1)+(1-\gamma)p\leq p\mathcal{I}(h_{0}\neq1)+w_{0}+T_{0}^{b}\mathcal{I}(h_{0}>1)
\]
\[
c_{1}+ph_{1}\mathcal{I}(h_{1}\neq h_{0})+(1-\gamma)ph_{0}\mathcal{I}(h_{0}\neq1)\leq ph_{0}\mathcal{I}(h_{1}\neq h_{0})+w_{1}+T_{1}^{b}\mathcal{I}(h_{1}>1)
\]
If $h_{0}=1$ then the first one is
\[
c_{0}+(1-\gamma)p\leq w_{0}+T_{0}^{b}\mathcal{I}(h_{0}>1)
\]
if $h_{0}\neq1$ then
\[
c_{0}+\gamma p(h_{0}-1)\leq w_{0}+T_{0}^{b}\mathcal{I}(h_{0}>1).
\]
The second if $h_{1}=h_{0}$ is
\[
c_{1}+(1-\gamma)ph_{0}\mathcal{I}(h_{0}\neq1)\leq w_{1}+T_{1}^{b}\mathcal{I}(h_{1}>1)
\]
else
\[
c_{1}+p(h_{1}-h_{0})+(1-\gamma)ph_{0}\mathcal{I}(h_{0}\neq1)\leq w_{1}+T_{1}^{b}\mathcal{I}(h_{1}>1)
\]
the second is equal to the first when $h_{1}=h_{0}$ therefore we
can use the conditions
\[
c_{0}+\gamma ph_{0}\mathcal{I}(h_{0}\neq1)+(1-\gamma)p\leq p\mathcal{I}(h_{0}\neq1)+w_{0}+T_{0}^{b}\mathcal{I}(h_{0}>1)
\]
\[
c_{1}+p(h_{1}-h_{0})+(1-\gamma)ph_{0}\mathcal{I}(h_{0}\neq1)\leq w_{1}+T_{1}^{b}\mathcal{I}(h_{1}>1)
\]
These two bind, and hence we can write
\[
C\equiv c_{0}+c_{1}=p\mathcal{I}(h_{0}\neq1)+w_{0}+T_{0}^{b}\mathcal{I}(h_{0}>1)-\gamma ph_{0}\mathcal{I}(h_{0}\neq1)-(1-\gamma)p
\]
\[
w_{1}+T_{1}^{b}\mathcal{I}(h_{1}>1)-p(h_{1}-h_{0})-(1-\gamma)ph_{0}\mathcal{I}(h_{0}\neq1)
\]
\[
=p(1-h_{0})\mathcal{I}(h_{0}\neq1)-p(h_{1}-h_{0})-(1-\gamma)p+w_{0}+w_{1}+T_{0}^{b}\mathcal{I}(h_{0}>1)+T_{1}^{b}\mathcal{I}(h_{1}>1)
\]
\[
=p(1-h_{0})-p(h_{1}-h_{0})-(1-\gamma)p+w_{0}+w_{1}+T_{0}^{b}\mathcal{I}(h_{0}>1)+T_{1}^{b}\mathcal{I}(h_{1}>1)
\]
\[
=p(1-h_{1})-(1-\gamma)p+w_{0}+w_{1}+T_{0}^{b}\mathcal{I}(h_{0}>1)+T_{1}^{b}\mathcal{I}(h_{1}>1)
\]
On top of these two constraints, the household also has the collateral
constraints
\[
\gamma ph_{0}\leq\gamma p+w_{0}+T_{0}^{c}\Leftrightarrow h_{0}\leq1+\frac{w_{0}+T_{0}^{c}}{\gamma p}
\]
\[
\gamma ph_{1}\leq\gamma ph_{0}\mathcal{I}(h_{0}\neq1)+p\mathcal{I}(h_{0}=1)+w_{1}+T_{1}^{c}\Leftrightarrow h_{1}\leq h_{0}\mathcal{I}(h_{0}\neq1)+\frac{\mathcal{I}(h_{0}=1)}{\gamma}+\frac{w_{1}+T_{1}^{c}}{\gamma p}
\]
This assymetry is due to the fact that if the household does not purchase
a house in period $0$ it will not be indebted to period $t+1$. Kind
of a consequence of this financial market...

\paragraph*{The full problem}

\[
\max_{C,h_{0},h_{1}}C+v(h_{1})\text{ s.to.}
\]
\[
C=p(1-h_{1})-(1-\gamma)p+w_{0}+w_{1}+T_{0}^{b}\mathcal{I}(h_{0}>1)+T_{1}^{b}\mathcal{I}(h_{1}>1)
\]
\[
h_{0}\leq1+\frac{w_{0}+T_{0}^{c}}{\gamma p}\text{ if \ensuremath{h_{0}\neq1}}
\]
\[
h_{1}\leq h_{0}\mathcal{I}(h_{0}\neq1)+\frac{\mathcal{I}(h_{0}=1)}{\gamma}+\frac{w_{1}+T_{1}^{c}}{\gamma p}\text{ if \ensuremath{h_{1}\neq h_{0}}}
\]


\paragraph*{The unconstrained problem
\[
\max_{C,h_{0},h_{1}}C+v(h_{1})\text{ s.to.}
\]
\[
C=p(1-h_{1})-(1-\gamma)p+w_{0}+w_{1}+T_{0}^{b}\mathcal{I}(h_{0}>1)+T_{1}^{b}\mathcal{I}(h_{1}>1)
\]
}

The marginal condition is
\[
v'(h_{1})=p
\]
This defines the unconstrained optimal level of housing
\[
h^{*}=[v']^{-1}(p)
\]
assume $h^{*}>1$. Assume that $T_{0}^{b},T_{1}^{b}\geq0$, then this
is the optimal choice.

\paragraph*{An effectively unconstrained household at time $0$}

Suppose that for a household
\[
1+\frac{w_{0}+T_{0}^{c}}{\gamma p}\geq h^{*}\Leftrightarrow w_{0}\geq\overline{w}\equiv\gamma p(h^{*}-1)-T_{0}^{c}.
\]
Then the household chooses 
\[
h_{0}=h_{1}=h^{*}
\]
\[
C=p(1-h^{*})-(1-\gamma)p+w_{0}+w_{1}+T_{0}^{b}+T_{1}^{b}
\]


\paragraph*{Slightly constrained household at time $0$}

Consider instead households such that
\[
w_{0}<\overline{w}.
\]
Define 
\[
\overline{h}_{0}=1+\frac{w_{0}+T_{0}^{c}}{\gamma p}
\]
which is the maximum house that the household could buy. The household
wants to get $h_{1}$ the closes it can to $h^{*}$, and since $h_{0}$
doesn't appear directly in the definition of $C$, conditional on
moving to a new house the household chooses $h_{0}=\overline{h}_{0}$.
Then the problem becomes
\[
\max_{C,h_{0},h_{1}}C+v(h_{1})\text{ s.to.}
\]
\[
C=p(1-h_{1})-(1-\gamma)p+w_{0}+w_{1}+T_{0}^{b}\mathcal{I}(h_{0}>1)+T_{1}^{b}\mathcal{I}(h_{1}>1)
\]
\[
h_{0}\in\{\overline{h}_{0},1\}
\]
\[
h_{1}\leq\overline{h}_{0}\mathcal{I}(h_{0}\neq1)+\frac{\mathcal{I}(h_{0}=1)}{\gamma}+\frac{w_{1}+T_{1}^{c}}{\gamma p}\text{ if \ensuremath{h_{1}\neq h_{0}}}
\]

\begin{claim*}
Suppose that
\[
\overline{h}_{0}\geq\frac{1}{\gamma}
\]
then the household moves in period 0. (I am actually assuming that
$h^{*}>1/\gamma$).
\end{claim*}
This is simple. If moving makes us less constrained tomorrow and also
gives the transfer today then it is strictly better to move.
\begin{claim*}
Suppose that 
\[
\overline{h}_{0}+\frac{w_{1}+T_{1}^{c}}{\gamma p}\geq h^{*}\text{ and \ensuremath{\overline{h}_{0}>1}}
\]
then it is also strictly better to move because the household gets
the transfer ans still goes to the optimal house.
\end{claim*}
If either 
\[
\overline{h}_{0}\geq\frac{1}{\gamma}\text{ or \ensuremath{\overline{h}_{0}+\frac{w_{1}+T_{1}^{c}}{\gamma p}\geq h^{*}}}
\]
then the household moves in period $0$ and in period $1$. These
constraints can be written as
\[
1+\frac{w_{0}+T_{0}^{c}}{\gamma p}\geq\frac{1}{\gamma}\Leftrightarrow w_{0}\geq p(1-\gamma)-T_{0}^{c}\Rightarrow\begin{cases}
h_{0}=1+\frac{w_{0}+T_{0}^{c}}{\gamma p}\\
h_{1}=\max\left\{ \bar{h}_{0},\min\left\{ h^{*},\overline{h}_{0}+\frac{w_{1}+T_{1}^{c}}{\gamma p}\right\} \right\} 
\end{cases}
\]

\[
w_{0}>-T_{0}^{c}
\]
\[
1+\frac{w_{0}+T_{0}^{c}}{\gamma p}+\frac{w_{1}+T_{1}^{c}}{\gamma p}\geq h^{*}\Leftrightarrow w_{1}\geq\gamma p(h^{*}-1)-T_{0}^{c}-T_{1}^{c}-w_{0}\Rightarrow\begin{cases}
h_{0}=1+\frac{w_{0}+T_{0}^{c}}{\gamma p}\\
h_{1}=h^{*}
\end{cases}
\]


\paragraph*{Very very constrained households}

Suppose that
\[
w_{0}\leq-T_{0}^{c}
\]
These households never buy in period $0$. They buy in period $1$
if they can, so
\[
h_{0}=1
\]
\[
h_{1}=\max\left\{ 1,\min\left\{ h^{*},\frac{1}{\gamma}+\frac{w_{1}+T_{1}^{c}}{\gamma p}\right\} \right\} 
\]
\[
=\begin{cases}
1 & \text{if \ensuremath{w_{1}\leq\gamma\left(1-\frac{1}{\gamma}\right)-T_{1}^{c}}}\\
\frac{1}{\gamma}+\frac{w_{1}+T_{1}^{c}-1}{\gamma p} & \text{if \ensuremath{w_{1}<\gamma p\left(h^{*}-\frac{1}{\gamma}\right)-T_{1}^{c}}}\\
h^{*} & \text{if \ensuremath{w_{1}>\gamma p\left(h^{*}-\frac{1}{\gamma}\right)-T_{1}^{c}}}
\end{cases}
\]


\paragraph*{Slightly more constrained households}

Suppose that
\[
-T_{0}^{c}<w_{0}<p(1-\gamma)-T_{0}^{c}
\]
and
\[
w_{1}<\gamma p(h^{*}-1)-T_{0}^{c}-T_{1}^{c}-w_{0}
\]
These are households that can buy houses in period $0$ but its not
necessarily optimal for them to do so, this is because by buying a
house they become more constrained.

If there was no transfer in case they move houses in period $0$ this
would be simple. They would not buy a house, and in period $1$ their
decisions would be the usual.

\subparagraph*{Very constrained households}

Suppose that
\[
w_{1}\leq\gamma p\left(1-\frac{1}{\gamma}\right)-T_{1}^{c}
\]
If they don't buy in time $0$ they cannot buy a house larger than
1 in time $1$. However, they can still purchase a larger house in
period $0$
\[
h_{0}=1+\frac{w_{0}+T_{0}^{c}}{\gamma p}
\]
\[
h_{1}=h_{0}
\]


\subparagraph*{With the transfer}

Assume that 
\[
w_{1}>\gamma p\left(1-\frac{1}{\gamma}\right)-T_{1}^{c}
\]
which just means that if the household does not purchase a house at
time $0$ it can do so at time $1$.

If the household buys a house it will buy
\[
\overline{h}_{0}=1+\frac{w_{0}+T_{0}^{c}}{\gamma p}\in\left(1,\frac{1}{\gamma}\right)
\]
and at time 1
\[
\tilde{h}_{1}=\overline{h}_{0}+\max\left\{ 0,\frac{w_{1}+T_{1}^{c}}{\gamma p}\right\} =1+\frac{w_{0}+T_{0}^{c}}{\gamma p}+\max\left\{ 0,\frac{w_{1}+T_{1}^{c}}{\gamma p}\right\} 
\]
With utility
\[
V^{p}=p(1-\tilde{h}_{1})-(1-\gamma)p+w_{0}+w_{1}+T_{0}^{b}+T_{1}^{b}+v(\tilde{h}_{1})
\]

Else, if the household does not buy in period $0$ then
\[
h_{0}=1
\]
\[
\hat{h}_{1}=\min\left\{ h^{*},\frac{1}{\gamma}+\frac{w_{1}+T_{1}^{c}}{\gamma p}\right\} >\tilde{h}_{1}
\]
With utility
\[
V^{n}=p(1-\hat{h}_{1})-(1-\gamma)p+w_{0}+w_{1}+T_{1}^{b}+v(\hat{h}_{1})
\]
Then the household purchases a house in period $1$ if
\[
V^{p}\geq V^{n}\Leftrightarrow p(\hat{h}_{1}-\tilde{h}_{1})+T_{0}^{b}\geq v(\hat{h}_{1})-v(\tilde{h}_{1})
\]


\subparagraph*{Suppose that $h^{*}\protect\leq\frac{1}{\gamma}+\frac{w_{1}+T_{1}^{c}}{\gamma p}\Leftrightarrow w_{1}\protect\geq\gamma p\left(h^{*}-\frac{1}{\gamma}\right)-T_{1}^{c}$}

Then
\[
V^{p}\geq V^{n}\Leftrightarrow v(\tilde{h}_{1})-p\tilde{h}_{1}+T_{0}^{b}\geq v(h^{*})-ph^{*}
\]
There is a single $h_{1}^{c}<h^{*}$ such that
\[
v(h_{1}^{c})-ph_{1}^{c}+T_{0}^{b}=v(h^{*})-ph^{*},
\]
Furthermore, the household will buy if they can afford more than this
level of housing, and will not buy in period $0$ if they can't.

Proof is simple: RHS is increasing in $h_{1}^{c}$ because $h_{1}^{c}<h^{*}$
furthermore, as $h_{1}^{c}\to h^{*}$ then the RHS is strictly larger
than the LHS. Assume that $h_{1}^{c}>1$. This means that the household
will buy if 
\[
1+\frac{w_{0}+T_{0}^{c}}{\gamma p}+\max\left\{ 0,\frac{w_{1}+T_{1}^{c}}{\gamma p}\right\} \geq h_{1}^{c}\Leftrightarrow w_{1}\geq\gamma p\left(h_{1}^{c}-1\right)-T_{0}^{c}-T_{1}^{c}-w_{0}
\]
Then if
\[
w_{1}\geq\gamma p\left(h^{*}-\frac{1}{\gamma}\right)-T_{1}^{c}\text{ and \ensuremath{w_{1}\geq\gamma p\left(h_{1}^{c}-1\right)-T_{0}^{c}-T_{1}^{c}-w_{0}}}
\]
The household acquires
\[
h_{0}=1+\frac{w_{0}+T_{0}^{c}}{\gamma p}
\]
\[
h_{1}=h_{0}+\frac{w_{1}+T_{1}^{c}}{\gamma p}
\]
If instead
\[
w_{1}\geq\gamma p\left(h^{*}-\frac{1}{\gamma}\right)-T_{1}^{c}\text{ and \ensuremath{w_{1}<\gamma p\left(h_{1}^{c}-1\right)-T_{0}^{c}-T_{1}^{c}-w_{0}}}
\]
Then
\[
h_{0}=1
\]
\[
h_{1}=h^{*}
\]
Furthermore,
\[
\frac{dh_{1}^{c}}{dT_{0}^{b}}\left(v'(h_{1}^{c})-p\right)+1=0\Leftrightarrow\frac{dh_{1}^{c}}{dT_{0}^{b}}=-\frac{1}{v'(h_{1}^{c})-p}<0
\]


\subparagraph*{Suppose that $w_{1}\protect\leq-T_{1}^{c}$}

Then
\[
\tilde{h}_{1}=1+\frac{w_{0}+T_{0}^{c}}{\gamma p}
\]
and
\[
\hat{h}_{1}=\frac{1}{\gamma}+\frac{w_{1}+T_{1}^{c}}{\gamma p}
\]
The condition for optimality is
\[
p(\hat{h}_{1}-\tilde{h}_{1})+T_{0}^{b}\geq v(\hat{h}_{1})-v(\tilde{h}_{1})
\]
\[
v\left(\tilde{h}_{1}\right)-p\tilde{h}_{1}+T_{0}^{b}\geq v\left(\frac{1}{\gamma}+\frac{w_{1}+T_{1}^{c}}{\gamma p}\right)-p\left(\frac{1}{\gamma}+\frac{w_{1}+T_{1}^{c}}{\gamma p}\right)
\]
\[
v\left(1+\frac{w_{0}+T_{0}^{c}}{\gamma p}\right)-p\left(1+\frac{w_{0}+T_{0}^{c}}{\gamma p}\right)+T_{0}^{b}\geq v\left(\frac{1}{\gamma}+\frac{w_{1}+T_{1}^{c}}{\gamma p}\right)-p\left(\frac{1}{\gamma}+\frac{w_{1}+T_{1}^{c}}{\gamma p}\right)
\]
This defines a relation, $w_{1}^{d}(w_{0})$ such that if $w_{1}<w_{1}^{d}(w_{0})$
it is worthwile to buy today, else if $w_{1}>w_{1}^{d}(w_{0})$ we
should wait to tomorrow to buy, and
\[
v\left(1+\frac{w_{0}+T_{0}^{c}}{\gamma p}\right)-p\left(1+\frac{w_{0}+T_{0}^{c}}{\gamma p}\right)+T_{0}^{b}=v\left(\frac{1}{\gamma}+\frac{w_{1}^{d}(w_{0};T)+T_{1}^{c}}{\gamma p}\right)-p\left(\frac{1}{\gamma}+\frac{w_{1}^{d}(w_{0};T)+T_{1}^{c}}{\gamma p}\right)
\]
this relation is increasing in $w_{0}$:
\[
v'\left(1+\frac{w_{0}+T_{0}^{c}}{\gamma p}\right)-p=\left(v'\left(\frac{1}{\gamma}+\frac{w_{1}^{d}(w_{0})+T_{1}^{c}}{\gamma p}\right)-p\right)\frac{dw_{1}^{d}(w_{0})}{dw_{0}}
\]
\[
\frac{dw_{1}^{d}(w_{0})}{dw_{0}}=\frac{v'\left(1+\frac{w_{0}+T_{0}^{c}}{\gamma p}\right)-p}{v'\left(\frac{1}{\gamma}+\frac{w_{1}^{d}(w_{0})+T_{1}^{c}}{\gamma p}\right)-p}>0
\]


\subparagraph*{Finally suppose that $w_{1}>-T_{1}^{c}$ and $w_{1}<\gamma p\left(h^{*}-\frac{1}{\gamma}\right)-T_{1}^{c}$}

If purchasing in period $0$ then
\[
h_{0}=1+\frac{w_{0}+T_{0}^{c}}{\gamma p}
\]
\[
\overline{h}_{1}=h_{0}+\frac{w_{1}+T_{1}^{c}}{\gamma p}=1+\frac{w_{0}+w_{1}+T_{1}^{c}+T_{0}^{c}}{\gamma p}
\]
which implies that
\[
V^{p}=p(1-\overline{h}_{1})-(1-\gamma)p+w_{0}+w_{1}+T_{0}^{b}+T_{1}^{b}+v(\overline{h}_{1})
\]
else if not buying in period $0$ we get
\[
h_{0}=1
\]
\[
\hat{h}_{1}=\frac{1}{\gamma}+\frac{w_{1}+T_{1}^{c}}{\gamma p}
\]
and
\[
V^{n}=p(1-\hat{h}_{1})-(1-\gamma)p+w_{0}+w_{1}+T_{1}^{b}+v(\hat{h}_{1})
\]
Note that $\overline{h}_{1}<\hat{h}_{1}.$ This implies that the household
will buy a house in period $0$ if
\[
V^{p}\geq V^{n}\Leftrightarrow v(\overline{h}_{1})-p\overline{h}_{1}+T_{0}^{b}\geq v(\hat{h}_{1})-p\hat{h}_{1}
\]
\[
v\left(1+\frac{w_{0}+w_{1}+T_{1}^{c}+T_{0}^{c}}{\gamma p}\right)-p\left(1+\frac{w_{0}+w_{1}+T_{1}^{c}+T_{0}^{c}}{\gamma p}\right)+T_{0}^{b}\geq v\left(\frac{1}{\gamma}+\frac{w_{1}+T_{1}^{c}}{\gamma p}\right)-p\left(\frac{1}{\gamma}+\frac{w_{1}+T_{1}^{c}}{\gamma p}\right)
\]
Fix $w_{0}$. The LHS grows faster than the RHS. There exists a single
$\hat{w}_{1}(w_{0})$ such that
\[
v\left(1+\frac{w_{0}+\hat{w}_{1}(w_{0})+T_{1}^{c}+T_{0}^{c}}{\gamma p}\right)-p\left(1+\frac{w_{0}+\hat{w}_{1}(w_{0})+T_{1}^{c}+T_{0}^{c}}{\gamma p}\right)+T_{0}^{b}=v\left(\frac{1}{\gamma}+\frac{\hat{w}_{1}(w_{0})+T_{1}^{c}}{\gamma p}\right)-p\left(\frac{1}{\gamma}+\frac{\hat{w}_{1}(w_{0})+T_{1}^{c}}{\gamma p}\right)
\]
and for $w_{1}\geq\hat{w}_{1}(w_{0})$ the household should buy, else
it should not buy. Furthermore, useful things
\[
\frac{1+\hat{w}_{1}'(w_{0})}{\gamma p}\left(v'\left(1+\frac{w_{0}+\hat{w}_{1}(w_{0})+T_{1}^{c}+T_{0}^{c}}{\gamma p}\right)-p\right)=\frac{\hat{w}_{1}'(w_{0})}{\gamma p}\left(v'\left(\frac{1}{\gamma}+\frac{\hat{w}_{1}(w_{0})+T_{1}^{c}}{\gamma p}\right)-p\right)
\]
\[
\hat{w}_{1}'(w_{0})=\frac{v'\left(\frac{1}{\gamma}+\frac{\hat{w}_{1}(w_{0})+T_{1}^{c}}{\gamma p}\right)-v'\left(1+\frac{w_{0}+\hat{w}_{1}(w_{0})+T_{1}^{c}+T_{0}^{c}}{\gamma p}\right)}{v'\left(1+\frac{w_{0}+\hat{w}_{1}(w_{0})+T_{1}^{c}+T_{0}^{c}}{\gamma p}\right)-p}<0
\]

The transfers are complicating everything, so let me do it without
the transfers...

\paragraph{Slightly more constrained households when $T_{0}^{b}=0$}

This case is nice because the household is not trading off getting
a transfer in period zero versus having a larger house tomorrow. It
is just about whatever maximizes the house level in period $1$.

\subparagraph*{Households very constrained at time 1}

Suppose that
\[
w_{1}\leq\gamma p\left(1-\frac{1}{\gamma}\right)-T_{1}^{c}
\]
If they don't buy in time $0$ they cannot buy a house larger than
1 in time $1$. However, they can still purchase a larger house in
period $0$
\[
h_{0}=1+\frac{w_{0}+T_{0}^{c}}{\gamma p}
\]
\[
h_{1}=h_{0}
\]


\subparagraph{Households not constrained at time 1}

Suppose that
\[
w_{1}\geq\gamma p\left(h^{*}-\frac{1}{\gamma}\right)-T_{1}^{c}
\]
this is the case in which the household can move to the optimal house
at time $1$ if it does not move in period $0$
\[
h_{0}=1
\]
then
\[
h_{1}=h^{*}
\]
which is optimal. Here I assume that households such that 
\[
w_{1}\geq\gamma p(h^{*}-1)-T_{0}^{c}-T_{1}^{c}-w_{0}
\]
also buy at time 0. In fact they are indifferent, since their time
$1$ house is $h^{*}$ in any case, but assume that if
\[
w_{1}\geq\gamma p(h^{*}-1)-T_{0}^{c}-T_{1}^{c}-w_{0}
\]
then
\[
h_{0}=1+\frac{w_{0}+T_{0}^{c}}{\gamma p}
\]
\[
h_{1}=h^{*}
\]


\subparagraph*{Households slightly constrained at time 1}

Suppose that
\[
w_{1}\geq-T_{1}^{c}
\]
this is the case in which the household can move to a larger house
that $1/\gamma$ if it does move in period 0, or larger than $h_{0}=1+\frac{w_{0}+T_{0}}{\gamma p}$,
since we are looking at the region in which
\[
1+\frac{w_{0}+T_{0}}{\gamma p}<\frac{1}{\gamma}\Leftrightarrow w_{0}<\gamma p\left(\frac{1}{\gamma}-1\right)-T_{0}^{c}
\]
then it is better not to change in period 0
\[
h_{0}=1
\]
\[
h_{1}=\frac{1}{\gamma}+\frac{w_{1}+T_{1}^{c}}{\gamma p}
\]


\subparagraph*{Finally the last case}

Suppose that
\[
\gamma p\left(1-\frac{1}{\gamma}\right)-T_{1}^{c}\leq w_{1}<-T_{1}^{c}
\]
then if the household buys a house in period $0$ it gets
\[
h_{1}=h_{0}=1+\frac{w_{0}+T_{0}^{c}}{\gamma p}
\]
If the household does not then
\[
h_{1}=\frac{1}{\gamma}+\frac{w_{1}+T_{1}^{c}}{\gamma p}
\]
Then the household buys in period $1$ if
\[
\frac{1}{\gamma}+\frac{w_{1}+T_{1}^{c}}{\gamma p}\leq1+\frac{w_{0}+T_{0}^{c}}{\gamma p}\Leftrightarrow w_{1}\leq\gamma p\left(1-\frac{1}{\gamma}\right)+T_{0}^{c}-T_{1}^{c}+w_{0}.
\]


\section*{Summary without transfers }

Without transfers we get the following table and figure

\includegraphics[scale=0.3]{Thresholds}
\begin{sidewaystable}[ph]
\centering{}%
\begin{tabular}{|c|c|c|c|}
\hline 
$w_{0}$ & $w_{1}$ & $h_{0}$ & $h_{1}$\tabularnewline
\hline 
\hline 
$\left(-\infty,-T_{0}^{c}\right)$ & $\left(-\infty,\gamma p\left(1-\frac{1}{\gamma}\right)-T_{1}^{c}\right)$ & $1$ & $1$\tabularnewline
\hline 
$\left(-\infty,-T_{0}^{c}\right)$ & $\left(\gamma p\left(1-\frac{1}{\gamma}\right)-T_{1}^{c},\gamma p\left(h^{*}-\frac{1}{\gamma}\right)-T_{1}^{c}\right)$ & $1$ & $\frac{1}{\gamma}+\frac{w_{1}+T_{1}^{c}}{\gamma p}$\tabularnewline
\hline 
$\left(-\infty,-T_{0}^{c}\right)$ & $\left(\gamma p\left(h^{*}-\frac{1}{\gamma}\right)-T_{1}^{c},+\infty\right)$ & $1$ & $h^{*}$\tabularnewline
\hline 
$\left(-T_{0}^{c},\gamma p\left(\frac{1}{\gamma}-1\right)-T_{0}^{c}\right)$ & $\left(-\infty,\gamma p\left(1-\frac{1}{\gamma}\right)-T_{1}^{c}\right)$ & $1+\frac{w_{0}+T_{0}^{c}}{\gamma p}$ & $1+\frac{w_{0}+T_{0}^{c}}{\gamma p}$\tabularnewline
\hline 
$\left(-T_{0}^{c},\gamma p\left(\frac{1}{\gamma}-1\right)-T_{0}^{c}\right)$ & $\left(\gamma p\left(1-\frac{1}{\gamma}\right)-T_{1}^{c},\gamma p\left(1-\frac{1}{\gamma}\right)+T_{0}^{c}-T_{1}^{c}+w_{0}\right)$ & $1+\frac{w_{0}+T_{0}^{c}}{\gamma p}$ & $1+\frac{w_{0}+T_{0}^{c}}{\gamma p}$\tabularnewline
\hline 
$\left(-T_{0}^{c},\gamma p\left(\frac{1}{\gamma}-1\right)-T_{0}^{c}\right)$ & $\left(\gamma p\left(1-\frac{1}{\gamma}\right)+T_{0}^{c}-T_{1}^{c}+w_{0},-T_{1}^{c}\right)$ & $1$ & $\frac{1}{\gamma}+\frac{w_{1}+T_{1}^{c}}{\gamma p}$\tabularnewline
\hline 
$\left(-T_{0}^{c},\gamma p\left(\frac{1}{\gamma}-1\right)-T_{0}^{c}\right)$ & $\left(-T_{1}^{c},\gamma p\left(h^{*}-\frac{1}{\gamma}\right)-T_{1}^{c}\right)$ & $1$ & $\frac{1}{\gamma}+\frac{w_{1}+T_{1}^{c}}{\gamma p}$\tabularnewline
\hline 
$\left(-T_{0}^{c},\gamma p\left(\frac{1}{\gamma}-1\right)-T_{0}^{c}\right)$ & $\left(\gamma p\left(h^{*}-\frac{1}{\gamma}\right)-T_{1}^{c},\gamma p(h^{*}-1)-T_{0}^{c}-T_{1}^{c}-w_{0}\right)$ & $1$ & $h^{*}$\tabularnewline
\hline 
$\left(-T_{0}^{c},\gamma p\left(\frac{1}{\gamma}-1\right)-T_{0}^{c}\right)$ & $\left(\gamma p(h^{*}-1)-T_{0}^{c}-T_{1}^{c}-w_{0}.+\infty\right)$ & $1+\frac{w_{0}+T_{0}^{c}}{\gamma p}$ & $h^{*}$\tabularnewline
\hline 
$\left(\gamma p\left(\frac{1}{\gamma}-1\right)-T_{0}^{c},\gamma p\left(h^{*}-1\right)-T_{0}^{c}\right)$ & $\left(-\infty,-T_{1}^{c}\right)$ & $1+\frac{w_{0}+T_{0}^{c}}{\gamma p}$ & $1+\frac{w_{0}+T_{0}^{c}}{\gamma p}$\tabularnewline
\hline 
$\left(\gamma p\left(\frac{1}{\gamma}-1\right)-T_{0}^{c},\gamma p\left(h^{*}-1\right)-T_{0}^{c}\right)$ & $\left(-T_{1}^{c},\gamma p(h^{*}-1)-T_{0}^{c}-T_{1}^{c}-w_{0}\right)$ & $1+\frac{w_{0}+T_{0}^{c}}{\gamma p}$ & $1+\frac{w_{0}+T_{0}^{c}}{\gamma p}+\frac{w_{1}+T_{1}^{c}}{\gamma p}$\tabularnewline
\hline 
$\left(\gamma p\left(\frac{1}{\gamma}-1\right)-T_{0}^{c},\gamma p\left(h^{*}-1\right)-T_{0}^{c}\right)$ & $\left(\gamma p(h^{*}-1)-T_{0}^{c}-T_{1}^{c}-w_{0}.+\infty\right)$ & $1+\frac{w_{0}+T_{0}^{c}}{\gamma p}$ & $h^{*}$\tabularnewline
\hline 
$\left(\gamma p\left(h^{*}-1\right)-T_{0}^{c},+\infty\right)$ & $-$ & $h^{*}$ & $h^{*}$\tabularnewline
\hline 
\end{tabular}
\end{sidewaystable}

\newpage 

Here goes a figure about what the change in $T_{0}^{c}$ does to those
thresholds. Continuous lines are the new thresholds, after $T_{0}^{c}\uparrow$.

\includegraphics[scale=0.3,bb = 0 0 200 100, draft, type=eps]{ThresholdChange.png}

\section*{Summary with transfers }

Without transfers we get the following table and figure where 
\[
w_{1}^{d}(w_{0};T):\text{ }v\left(1+\frac{w_{0}+T_{0}^{c}}{\gamma p}\right)-p\left(1+\frac{w_{0}+T_{0}^{c}}{\gamma p}\right)+T_{0}^{b}=v\left(\frac{1}{\gamma}+\frac{w_{1}^{d}(w_{0};T)+T_{1}^{c}}{\gamma p}\right)-p\left(\frac{1}{\gamma}+\frac{w_{1}^{d}(w_{0};T)+T_{1}^{c}}{\gamma p}\right)
\]
\begin{align*}
\hat{w}_{1}(w_{0};T) & :\text{ \ensuremath{v\left(1+\frac{w_{0}+\hat{w}_{1}(w_{0};T)+T_{1}^{c}+T_{0}^{c}}{\gamma p}\right)-p\left(1+\frac{w_{0}+\hat{w}_{1}(w_{0};T)+T_{1}^{c}+T_{0}^{c}}{\gamma p}\right)+T_{0}^{b}}}\\
 & =v\left(\frac{1}{\gamma}+\frac{\hat{w}_{1}(w_{0};T)+T_{1}^{c}}{\gamma p}\right)-p\left(\frac{1}{\gamma}+\frac{\hat{w}_{1}(w_{0};T)+T_{1}^{c}}{\gamma p}\right)
\end{align*}
\[
h_{1}^{c}(T_{0}^{b}):\text{ \ensuremath{v(h_{1}^{c})-ph_{1}^{c}+T_{0}^{b}=v(h^{*})-ph^{*}}}
\]

\begin{sidewaystable}[ph]
\centering{}%
\begin{tabular}{|c|c|c|c|}
\hline 
$w_{0}$ & $w_{1}$ & $h_{0}$ & $h_{1}$\tabularnewline
\hline 
\hline 
$\left(-\infty,-T_{0}^{c}\right)$ & $\left(-\infty,\gamma p\left(1-\frac{1}{\gamma}\right)-T_{1}^{c}\right)$ & $1$ & $1$\tabularnewline
\hline 
$\left(-\infty,-T_{0}^{c}\right)$ & $\left(\gamma p\left(1-\frac{1}{\gamma}\right)-T_{1}^{c},\gamma p\left(h^{*}-\frac{1}{\gamma}\right)-T_{1}^{c}\right)$ & $1$ & $\frac{1}{\gamma}+\frac{w_{1}+T_{1}^{c}}{\gamma p}$\tabularnewline
\hline 
$\left(-\infty,-T_{0}^{c}\right)$ & $\left(\gamma p\left(h^{*}-\frac{1}{\gamma}\right)-T_{1}^{c},+\infty\right)$ & $1$ & $h^{*}$\tabularnewline
\hline 
$\left(-T_{0}^{c},\gamma p\left(\frac{1}{\gamma}-1\right)-T_{0}^{c}\right)$ & $\left(-\infty,\gamma p\left(1-\frac{1}{\gamma}\right)-T_{1}^{c}\right)$ & $1+\frac{w_{0}+T_{0}^{c}}{\gamma p}$ & $1+\frac{w_{0}+T_{0}^{c}}{\gamma p}$\tabularnewline
\hline 
$\left(-T_{0}^{c},\gamma p\left(\frac{1}{\gamma}-1\right)-T_{0}^{c}\right)$ & $\left(\gamma p\left(1-\frac{1}{\gamma}\right)-T_{1}^{c},w_{1}^{d}(w_{0};T)\right)$ & $1+\frac{w_{0}+T_{0}^{c}}{\gamma p}$ & $1+\frac{w_{0}+T_{0}^{c}}{\gamma p}$\tabularnewline
\hline 
$\left(-T_{0}^{c},\gamma p\left(\frac{1}{\gamma}-1\right)-T_{0}^{c}\right)$ & $\left(w_{1}^{d}(w_{0};T),-T_{1}^{c}\right)$ & $1$ & $\frac{1}{\gamma}+\frac{w_{1}+T_{1}^{c}}{\gamma p}$\tabularnewline
\hline 
$\left(-T_{0}^{c},\gamma p\left(\frac{1}{\gamma}-1\right)-T_{0}^{c}\right)$ & $\left(-T_{1}^{c},\hat{w}_{1}(w_{0};T)\right)$ & $1$ & $\frac{1}{\gamma}+\frac{w_{1}+T_{1}^{c}}{\gamma p}$\tabularnewline
\hline 
$\left(-T_{0}^{c},\gamma p\left(\frac{1}{\gamma}-1\right)-T_{0}^{c}\right)$ & $\left(\hat{w}_{1}(w_{0};T),\gamma p\left(h^{*}-\frac{1}{\gamma}\right)-T_{1}^{c}\right)$ & $1+\frac{w_{0}+T_{0}^{c}}{\gamma p}$ & $1+\frac{w_{0}+T_{0}^{c}}{\gamma p}+\frac{w_{1}+T_{1}^{c}}{\gamma p}$\tabularnewline
\hline 
$\left(-T_{0}^{c},\gamma p\left(\frac{1}{\gamma}-1\right)-T_{0}^{c}\right)$ & $\left(\gamma p\left(h^{*}-\frac{1}{\gamma}\right)-T_{1}^{c},\gamma p\left(h_{1}^{c}(T_{0}^{b})-1\right)-T_{0}^{c}-T_{1}^{c}-w_{0}\right)$ & $1$ & $h^{*}$\tabularnewline
\hline 
$\left(-T_{0}^{c},\gamma p\left(\frac{1}{\gamma}-1\right)-T_{0}^{c}\right)$ & $\left(\gamma p\left(h_{1}^{c}(T_{0}^{b})-1\right)-T_{0}^{c}-T_{1}^{c}-w_{0},\gamma p(h^{*}-1)-T_{0}^{c}-T_{1}^{c}-w_{0}\right)$ & $1+\frac{w_{0}+T_{0}^{c}}{\gamma p}$ & $1+\frac{w_{1}+T_{1}^{c}}{\gamma p}$\tabularnewline
\hline 
$\left(-T_{0}^{c},\gamma p\left(\frac{1}{\gamma}-1\right)-T_{0}^{c}\right)$ & $\left(\gamma p(h^{*}-1)-T_{0}^{c}-T_{1}^{c}-w_{0}.+\infty\right)$ & $1+\frac{w_{0}+T_{0}^{c}}{\gamma p}$ & $h^{*}$\tabularnewline
\hline 
$\left(\gamma p\left(\frac{1}{\gamma}-1\right)-T_{0}^{c},\gamma p\left(h^{*}-1\right)-T_{0}^{c}\right)$ & $\left(-\infty,-T_{1}^{c}\right)$ & $1+\frac{w_{0}+T_{0}^{c}}{\gamma p}$ & $1+\frac{w_{0}+T_{0}^{c}}{\gamma p}$\tabularnewline
\hline 
$\left(\gamma p\left(\frac{1}{\gamma}-1\right)-T_{0}^{c},\gamma p\left(h^{*}-1\right)-T_{0}^{c}\right)$ & $\left(-T_{1}^{c},\gamma p(h^{*}-1)-T_{0}^{c}-T_{1}^{c}-w_{0}\right)$ & $1+\frac{w_{0}+T_{0}^{c}}{\gamma p}$ & $1+\frac{w_{0}+T_{0}^{c}}{\gamma p}+\frac{w_{1}+T_{1}^{c}}{\gamma p}$\tabularnewline
\hline 
$\left(\gamma p\left(\frac{1}{\gamma}-1\right)-T_{0}^{c},\gamma p\left(h^{*}-1\right)-T_{0}^{c}\right)$ & $\left(\gamma p(h^{*}-1)-T_{0}^{c}-T_{1}^{c}-w_{0}.+\infty\right)$ & $1+\frac{w_{0}+T_{0}^{c}}{\gamma p}$ & $h^{*}$\tabularnewline
\hline 
$\left(\gamma p\left(h^{*}-1\right)-T_{0}^{c},+\infty\right)$ & $-$ & $h^{*}$ & $h^{*}$\tabularnewline
\hline 
\end{tabular}
\end{sidewaystable}

\newpage 

Here goes a figure about what the change in $T_{0}^{c}$ does to those
thresholds. Continuous lines are the new thresholds, after $T_{0}^{c}\uparrow$.
\end{document}
